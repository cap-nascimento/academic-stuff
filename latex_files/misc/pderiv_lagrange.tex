\documentclass[11pt]{article}
\usepackage[brazil]{babel}
\usepackage[utf8]{inputenc}
\usepackage[T1]{fontenc}
\usepackage{graphicx}
\usepackage{amsmath}
\usepackage{amssymb}

\begin{document}

\title{Lista de Exercícios - Prova 3 - 1/2018}
\author{Anderson Vieira do Nascimento}
\date{11/11/2018}
\maketitle

\section{Derivadas parciais em função descontínua}

Sejam o ponto $p = (0, 0)$ e a seguinte função $f(x, y)$:\\

\[ \begin{cases}
		\frac{\sqrt{x^2 + y^2} - x - y}{x^2 + y^2}, \ se \ (x, y) \neq (0, 0)\\
		0, \ se \ (x, y) = (0, 0)
	\end{cases}
\]\\

Temos as seguintes derivadas parciais em $p$:\\
\\
$f_x(0, 0):$
$$f_x(0, 0) = \lim_{h \rightarrow 0} \frac{\frac{\sqrt{(x+h)^2 + y^2} - (x + h) - y}{(x+h)^2 + y^2} - \frac{\sqrt{x^2 + y^2} - x - y}{x^2 + y^2}}{h}$$

$$f_x(0, 0) = \lim_{h \rightarrow 0} \frac{\frac{\sqrt{h^2} - h}{h^2} - 0}{h} = \lim_{h \rightarrow 0} \frac{0}{h^3} = 0$$\\
\\
$f_y(0, 0):$
$$f_y(0, 0) = \lim_{k \rightarrow 0} \frac{\frac{\sqrt{x^2 + (y+k)^2} - x - (y+k)}{x^2 + (y+k)^2} - \frac{\sqrt{x^2 + y^2} - x - y}{x^2 + y^2}}{k}$$

$$f_y(0, 0) = \lim_{k \rightarrow 0} \frac{\frac{\sqrt{k^2} - k}{k^2} - 0}{k} = \lim_{k \rightarrow 0} \frac{0}{k^3} = 0$$\\
\\
Temos que $f_x(0, 0) = f_y(0, 0) = 0$, logo as derivadas parciais existem mas $f$ não é contínua no ponto $p$.

\section{Diferenciabilidade de função em um ponto}

Pelo teorema:

\textit{Se as derivadas parciais $f_x$ e $f_y$ forem contínuas no ponto $p_0$, a função é diferenciável em $p_0$.}\\

Justificativa: \textit{Com derivadas contínuas é possível aproximar o valor de $f$ em $p_0$ com um plano tangente definido como:}

\begin{equation}
f(x, y) = f(x_0, y_0) + f_x(x_0, y_0)\Delta x + f_y(x_0, y_0)\Delta y
\end{equation}

\textit{Sendo possível fazer tal aproximação, a função é diferenciável.}

\section{Máximos locais, mínimos locais e pontos de sela}

Seja a função $f(x, y) = 2x^3 + 2y^3 - 9x^2 + 3y^2 - 12y$ e suas derivadas parciais:

\begin{equation}
f_x = 6x^2 - 18x
\end{equation}

\begin{equation}
f_y = 6y^2 + 6y - 12
\end{equation}

Temos que em (2), $f_x = 0$ se:

$$6x^2 - 18x = 0$$
$$x(6x - 18) = 0$$

Portanto, $x = 0$ ou $(6x - 18) = 0$, onde:

$$6x = 18$$
$$x = 3$$

Logo $x = \{0, 3\}$.\\

Temos que em (3), $f_y = 0$ se:

$$6y^2 + 6y - 12 = 0$$

Resolvendo pela fórmula de bhaskara, temos:

$$\Delta = \sqrt{b^2 - 4ac} = \sqrt{6^2 - 4 \cdot 6 \cdot(-12)} = \sqrt{324} = \pm 18$$

$$y = \frac{-6 \pm 18}{12} = \{-2, 1\}$$

Assim, temos quatro pontos críticos:

$$(0, 1), (0, -2), (3, 1), (3, -2)$$

Tomando a matriz Hessiana:\\

\[
H = 
\begin{bmatrix}
	f_{xx} & f_{xy}\\
	f_{yx} & f_{yy}
\end{bmatrix}
\]

\[
H = 
\begin{bmatrix}
	12x-18 & 0\\
	0      & 12y+6
\end{bmatrix}
\]

Para o ponto $(0, 1)$:

$$Det_M = (12 \cdot 0 - 18)\cdot(12 \cdot 1 + 6) - 0 = (-18)\cdot(18) = -324$$

Como $Det_M < 0$, temos que $(0, 1)$ é ponto de sela.\\
\\

Para o ponto $(0, -2)$:

$$Det_M = (12 \cdot 0 - 18)\cdot(12 \cdot (-2) + 6) - 0 = (-18)\cdot(-18) = 324$$

Como $Det_M > 0$ e $f_{xx} = -18 < 0$, temos que $(0, -2)$ é ponto de máximo local.\\
\\

Para o ponto $(3, 1)$:

$$Det_M = (12 \cdot 3 - 18)\cdot(12 \cdot 1 + 6) - 0 = (18)\cdot(18) = 324$$

Como $Det_M > 0$ e $f_{xx} = 18 > 0$, temos que $(3, 1)$ é ponto de mínimo local.\\
\\

Para o ponto $(3, -2)$:

$$Det_M = (12 \cdot 3 - 18)\cdot(12 \cdot (-2) + 6) - 0 = (18)\cdot(-18) = -324$$

Como $Det_M < 0$, temos que $(3, -2)$ é ponto de sela.\\
\\

\section{Regra da Cadeia}

Seja $w = f(x, y)$, $x = r \cdot cos(\theta)$ e $y = r \cdot sen(\theta)$:\\
\\
Obter $\frac{\partial w}{\partial r}$:\\
\\
Pela regra da cadeia, temos que:

$$\frac{\partial w}{\partial r} = \frac{\partial f}{\partial x} \cdot \frac{\mathrm{d} x}{\mathrm{d} r} + \frac{\partial f}{\partial y} \cdot \frac{\mathrm{d} y}{\mathrm{d} r}$$

$$\frac{\partial w}{\partial r} = \frac{\partial f}{\partial x} \cdot cos(\theta) + \frac{\partial f}{\partial y} \cdot sen(\theta)$$
\\
Obter $\frac{\partial w}{\partial r}$:\\
\\
Pela regra da cadeia, temos que:

$$\frac{\partial w}{\partial \theta} = \frac{\partial f}{\partial x} \cdot \frac{\mathrm{d} x}{\mathrm{d} \theta} + \frac{\partial f}{\partial y} \cdot \frac{\mathrm{d} y}{\mathrm{d} \theta}$$

$$= \frac{\partial f}{\partial x} \cdot (r \cdot (-sen(\theta))) + \frac{\partial f}{\partial y} \cdot (r \cdot cos(\theta))$$

$$= r \cdot \left(\frac{\partial f}{\partial x} \cdot (-sen(\theta)) + \frac{\partial f}{\partial y} \cdot (cos(\theta))\right)$$

Logo:

$$ \frac{1}{r} \cdot \frac{\partial w}{\partial \theta} = \frac{\partial f}{\partial x} \cdot (-sen(\theta)) + \frac{\partial f}{\partial y} \cdot (cos(\theta))$$

\section{Pontos críticos em curvas}

Seja $f(x, y) = x^2 - y^2$ e a parametrização $\alpha(t) = (2cos(t), 2sen(t))$, determinar os pontos críticos sobre a curva:\\
\\
Primeiro determinamos o vetor gradiente $\nabla f$:\\
\\
$$\nabla f = (f_x, f_y) = (2x, -2y)$$
\\
Depois, o vetor tangente à curva $\alpha'(t)$:\\
\\
$$\alpha'(t) = \left(\frac{\mathrm{d} x}{\mathrm{d} t}, \frac{\mathrm{d} y}{\mathrm{d} t}\right) = (-2sen(t), 2cos(t))$$
\\
Agora, os pontos onde os vetores forem ortogonais são pontos críticos na curva. Em outras palavras:\\
\\
$$\langle \nabla f, \alpha'(t) \rangle = 0$$\\
\\
Sendo $\nabla f$ expressa em função da curva, temos:\\
\\
$$\nabla f (\alpha(t)) = (2 \cdot (2cos(t)), -2 \cdot (2sen(t))) = (4cos(t), -4sen(t))$$\\
\\
Assim:\\
\\
$$\langle \nabla f, \alpha'(t) \rangle = 4cos(t) \cdot (-2sen(t)) + (-4sen(t)) \cdot (2cos(t))$$

$$\langle \nabla f, \alpha'(t) \rangle = -8cos(t) \cdot sen(t) - 8sen(t) \cdot cos(t) = 0$$

$$-16cos(t) \cdot sen(t) = 0$$

Temos que $-16cos(t) \cdot sen(t) = 0$ se $t = \{ 0, \frac{\pi}{2}, \pi, \frac{3\pi}{2}\}$\\
\\
Logo, os pontos onde a variação do consumo é nula são:\\
$$(2cos(0), 2sen(0)) = (2, 0)$$
$$(2cos(\frac{\pi}{2}), 2sen(\frac{\pi}{2})) = (0, 2)$$
$$(2cos(\pi), 2sen(\pi)) = (-2, 0)$$
$$(2cos(\frac{3\pi}{2}\}), 2sen(\frac{3\pi}{2}\})) = (0, -2)$$\\

Porém, os pontos onde o consumo é $4$ são apenas:\\
$$(2cos(0), 2sen(0)) = (2, 0)$$
$$(2cos(\pi), 2sen(\pi)) = (-2, 0)$$
\\
As direções dos pontos são dadas por $\alpha'(t)$:\\
$$\alpha'(t) = (-2sen(t), 2cos(t))$$
$$(-2sen(0), 2cos(0)) = (0, 2)$$
$$(-2sen(\pi), 2cos(\pi)) = (0, -2)$$\\

\section{Valores extremos por parametrização e multiplicadores de Lagrange}

Seja $f(x, y) = x + y$ e o círculo unitário $x^2 + y^2 = 1$, determinar os valores extremos da função sobre o círculo.\\
\\
\textit{Parametrização em sentido anti-horário}\\
\\
$$\alpha(t) = (-sen(t), cos(t))$$
Tomando $f(\alpha(t))$:
$$f(\alpha(t)) = (-sen(t)) + cos(t) = cos(t) - sen(t)$$
Assim:
$$\frac{\mathrm{d} f}{\mathrm{d} t} = -sen(t) - cos(t) = 0$$
Temos que $-sen(t) - cos(t) = 0 \Leftrightarrow -sen(t) = cos(t)$, logo $t = \{ \frac{3\pi}{4}, \frac{7\pi}{4} \}$. Os pontos críticos são:

$$\left(-sen \left(\frac{3\pi}{4} \right), cos \left(\frac{3\pi}{4} \right) \right) = \left( - \frac{\sqrt{2}}{2}, - \frac{\sqrt{2}}{2}\right)$$

$$\left(-sen \left(\frac{7\pi}{4} \right), cos \left(\frac{7\pi}{4} \right) \right) = \left( \frac{\sqrt{2}}{2}, \frac{\sqrt{2}}{2}\right)$$

Os valores extremos são:

$$f\left( - \frac{\sqrt{2}}{2}, - \frac{\sqrt{2}}{2} \right) = - \sqrt{2}$$
$$f\left( \frac{\sqrt{2}}{2}, \frac{\sqrt{2}}{2} \right) = \sqrt{2}$$

\textit{Pelo método dos multiplicadores de Lagrange}\\
\\
Temos o seguinte sistema:\\
\[
	\begin{cases}
		\nabla f = \lambda \nabla g\\
		g(x, y) = 0
	\end{cases}
\]
\\
Temos que $g(x, y) = x^2 + y^2 - 1$, assim:
$$\nabla f = (f_x, f_y) = (1, 1)$$
$$\nabla g = (g_x, g_y) = (2x, 2y)$$\\
Logo:
\[
	\begin{cases}
		(1, 1) = \lambda (2x, 2y) \ \ (I)\\
		x^2 + y^2 - 1 = 0 \ \ (II)
	\end{cases}
\]
\\
Da equação I, temos outro sistema:\\
\[
	\begin{cases}
		1 = \lambda 2x \\
		1 = \lambda 2y
	\end{cases}
\]\\
Que nos dá $x = y = \frac{1}{2\lambda}$. Substituindo os valores na equação II:\\
$$\left( \frac{1}{2\lambda} \right)^2 + \left( \frac{1}{2\lambda} \right)^2 = 1$$
$$ \frac{1}{4} \left( \frac{2}{\lambda^2} \right) = 1$$
$$ \left( \frac{1}{\lambda^2} \right) = 2$$
$$\lambda^2 = \frac{1}{2}$$
$$\lambda = \pm \frac{1}{\sqrt{2}}$$
Assim, se $\lambda = \pm \frac{1}{\sqrt{2}}$:

$$x = \frac{1}{2} \cdot (\pm \sqrt{2}) = \pm \frac{\sqrt{2}}{2}$$
$$y = \frac{1}{2} \cdot (\pm \sqrt{2}) = \pm \frac{\sqrt{2}}{2}$$

Valores para os quais temos valores extremos nos pontos:

$$f\left( - \frac{\sqrt{2}}{2}, - \frac{\sqrt{2}}{2} \right) = - \sqrt{2}$$
$$f\left( \frac{\sqrt{2}}{2}, \frac{\sqrt{2}}{2} \right) = \sqrt{2}$$

\end{document}